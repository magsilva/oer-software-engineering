\begin{frame}[parent={ie:agenda}, hasnext=false, hasprev=false]
	\frametitle{Engenharia de Software}

	\begin{block:concept}{Engenharia de Software}
		Disciplina cujo objetivo é a produção de software isento de falhas, \textbf{entregue
		dentro de prazo e orçamento previstos}, e que atenda às necessidades do
		cliente~\cite[p.~4]{Schach:2008}.
	\end{block:concept}
\end{frame}


\begin{frame}[hasnext=false, hasprev=true]
	\frametitle{Engenharia de Software}

	\begin{block:concept}{Engenharia de Software}
		Disciplina cujo objetivo é a produção de software isento de falhas, \textbf{entregue
		dentro de prazo e orçamento previstos}, \textbf{e que atenda às necessidades do
		cliente}~\cite[p.~4]{Schach:2008}.
	\end{block:concept}
	
	\begin{block:fact}{Atender necessidades do cliente}
		\begin{itemize}
			\item Qual é a responsabilidade do \textbf{projeto} quanto a satisfazer o cliente?
		\end{itemize}
	\end{block:fact}
	
	\note{
		Por exemplo, eu poderia construir uma ponte sem falhas, no prazo e orçamento previstos,
		ligando nada a lugar nenhum.
		\begin{itemize}
			\item Projeto ok.
			\item Produto: not ok.
		\end{itemize}

		Posso criar um software com poucos erros, no prazo e orçamento previstos, mas que:
		\begin{itemize}
			\item vendeu poucas unidades (COTS),
			\item foi produto para um cliente apenas (nenhum outro projeto futuro continou este software)
		\end{itemize}
	}
\end{frame}
