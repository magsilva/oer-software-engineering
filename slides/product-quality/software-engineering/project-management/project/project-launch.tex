\begin{frame}[parent={ie:agenda}, hasnext=true, hasprev=false]
	\frametitle{Produto e Projetos}

	\begin{block:fact}{Criação de um projeto para um produto}
		\begin{itemize}
			\item Elaboração de termo de abertura
			\item Elaboração de declaração de escopo
		\end{itemize}
	\end{block:fact}
\end{frame}


\begin{frame}[hasnext=true,hasprev=true]
	\frametitle{Termo de abertura}
	
	\begin{block:concept}{Termo de abertura}
		\begin{itemize}
			\item Objetivo e justificativa do projeto
			\item Descrição do produto
			\item Requisitos de alto nível
			\item Nomeação do gerente do projeto
			\item Cronograma de marcos resumido
			\item Definição de papéis e responsabilidades
			\item Premissas e hipóteses
			\item Restrições
			\item Orçamento previsto
		\end{itemize}
	\end{block:concept}
	
	\note{
		O termo de abertura confere ao gerente de projeto autoridade para iniciar
		de fato o projeto e a autonomia para gerí-lo.
	}
\end{frame}


\begin{frame}
	\frametitle{Declaração de escopo}
	
	\begin{block:concept}{Declaração de escopo}
		\begin{itemize}
			\item Descrição do produto
			\item Entregáveis (\foreign{deliverables})
			\item Objetivos do projeto
			\item Critérios de aceitação do produto
		\end{itemize}
	\end{block:concept}
	
	\note{
		Objetivos do projeto são quantificáveis. Eles são utilizados para determinar
		se o produto poderá ser aceito ou não.
	}
\end{frame}


% \begin{frame}
% 	\frametitle{Declaração de escopo}
% 	\framesubtitle{Objetivos do projeto e critérios de aceitação do produto}
%  
% 	\begin{block:fact}{Qualidade do produto}
% 		\begin{itemize}
% 			\item \textbf{Atributos de qualidade em uso}
% 			\item Atributos de qualidade externa
% 			\item Atributos de qualidade interna
% 		\end{itemize}
% 	\end{block:fact}
% 
% 
% 	\note{
% 		Medições de métrica de previsão (qualidade de produto de software)
% 
% 		Características desejáveis de um métrica de previsão:
% 		\begin{itemize}
% 			\item Atributo interno deve ser passível de medição com precisão
% 			\item Correlação entre atributo de qualidade interna e o atributo de
% 			qualidade externa
% 			\item Relação deve ser modelada matematicamente
% 		\end{itemize}
% 	}
% \end{frame}
