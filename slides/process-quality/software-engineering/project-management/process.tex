\begin{frame}[parent={ie:agenda}, hasnext=true, hasprev=true]
	\frametitle{Processo de software}

	\begin{block:concept}{Definição}
		Processos de software definem atividades e artefatos para
		a engenharia (desenvolvimento sistemático e repetível com qualidade) de
		software.
	\end{block:concept}
	
	\begin{block:fact}{Tópicos abordados}
		\begin{itemize}
			\item Definição de processos
			\item Melhoria de processo
		\end{itemize}
	\end{block:fact}
\end{frame}



\begin{frame}[parent={ie:agenda}, hasnext=true, hasprev=true]
	\frametitle{Processo de software}
	
	\begin{block:concept}{Elementos do processo de software}
		\begin{itemize}
			\item Processo: coleção	organizada de atividades.
			\item Atividade: conjunto de tarefas que levam a um ou mais artefatos de	qualidade controlada;
			\item Tarefa: ação desempenhada por algum papel visando a realização ou
				monitoramento do projeto;
			\item Papel: descreve como as pessoas atuam no processo e quais as suas
				responsabilidades;
			\item Artefato: resultado de uma atividade;
		\end{itemize}
	\end{block:concept}
\end{frame}

