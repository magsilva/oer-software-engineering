\begin{frame}[parent={ie:agenda}, hasnext=true, hasprev=false]
	\frametitle{Modelo incremental}

	\begin{block:concept}{Definição}
		Modelo de ciclo de vida organizado em iterações, com entrega de versões do
		produto a cada iteração, e em que o usuário participa em todas as
		etapas de desenvolvimento~\cite{Zurcher-Randel:1968, Basili-Turner:1975, Gilb:1981, McCracken-Jackson:1982, Gilb:1985}.
	\end{block:concept}
	
	\begin{block:fact}{Contexto}
		\begin{itemize}
			\item 1970 -- 1985
			\begin{itemize}
				\item Na verdade, remonta de 1930 e, para software, de 1960~\cite{Larman-Basili:2003}.
			\end{itemize}
			
			\item Resposta a um evento que pretendia estabelecer \textbf{o} ciclo de
			vida para desenvolvimento de software (para fins de educação e certificação).
		\end{itemize}
	\end{block:fact}
	
	\note{
		\begin{itemize}
			\item Apresentado originalmente em 1981, durante evento organizado pela
			ICCP e patrocinado pela IEEE-CS e ACM, com o título ``A Minority Dissenting
			Position''.
			
			\item O artigo foi uma resposta (e crítica) ao modelo de ciclo de vida de
			software adotado como ``correto'' pela conferência.
			
			\item Artigo é curto e superficial. Além disso, não oferece nenhuma
			evidência que suporta o modelo proposto.
			
			\item Mais sobre a história do modelo incremental: http://c2.com/cgi-bin/wiki?HistoryOfIterative
		\end{itemize}
	}
\end{frame}


\begin{frame}[hasnext=true, hasprev=true]
	\frametitle{Modelo incremental}

	\begin{block:fact}{Características}
		\begin{itemize}
			\item Dirigido a objetivos
			\begin{itemize}
				\item Requisitos de qualidade em uso (e mensuráveis)
				\item Objetivos revisados a cada iteração
			\end{itemize}
		
			\item Iterações frequentes
			\begin{itemize}
				\item Favorecimento de uma maior relação valor entregue para o usuário
				em relação ao custo de desenvolvimento
			\end{itemize}
			
			\item Análise, projeto, entrega e teste completos em cada iteração
		\end{itemize}
	\end{block:fact}
	
	\note{
		\begin{itemize}
			\item Ao invés de verificar o quanto podemos fazer com uma certa quantidade
			de recurso (tal como no planejamento usual), a pergunta é: quão pouco podemos
			gastar e ainda conseguir algum resultado útil para alcançar nossos objetivos?
		\end{itemize}
	}
\end{frame}



\begin{frame}[hasnext=false, hasprev=true]
	\frametitle{Modelo incremental}

	\begin{block:fact}{Pontos positivos}
		\begin{itemize}
			\item Foco nos interesses do usuário
		\end{itemize}
	\end{block:fact}

	\begin{block:fact}{Limitações}
		\begin{itemize}
			\item Risco elevado das iterações iniciais~\cite{Boehm:1996}
			\begin{itemize}
				\item Não atender um objetivo razoável do interessado.
				\item Requisitos de qualidade não são priorizados.
				\item Arquitetura inicial pode não ser satisfatória.
			\end{itemize}
		\end{itemize}
	\end{block:fact}
	
	\note{
		As limitações deste modelo inspiraram a definição de modelos baseados em risco
		por Barry Boehm (modelo espiral).
	}
\end{frame}