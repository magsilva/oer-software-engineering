\begin{frame}[parent={ie:agenda}, hasnext=true, hasprev=false]
	\frametitle{Considerações finais}

	
	\begin{block:fact}{Atualmente\ldots}
		\begin{itemize}
			\item Integração entre desenvolvimento de software e sistema~\cite{Boehm-Lane:2010}.
			
			\item Consolidação de modelos ágeis: iterativos, incrementais e centrados no usuário.
			
			\item Integração de métodos:
			\begin{itemize}
				\item Application Life-cycle Management (ALM)
				\item Continuous *
				\item DevOps (Development + Operation)
			\end{itemize}
		\end{itemize}
	\end{block:fact}
	
	\begin{block:fact}{Limitações}
		\begin{itemize}
			\item Poucos trabalhos avaliam empiricamente os diferentes tipos de
			ciclo de vida~\cite{Benediktsson-etal:2006}.
			
			\item One size does not fit all.
		\end{itemize}
	\end{block:fact}
	
	\note{
		\begin{itemize}
			\item Mais de 30 anos para adotar modelos iterativos\ldots
			
			\item Ciclos rápidos.
			
			\item DevOps (Desenvolvimento + Operacionalização)
			\begin{itemize}
				\item Continuous  planning,  collaborative  and  continuous  development,
				continuous  integration	and testing, continuous release and deployment,
				continuous infrastructure monitoring and optimization, continuous user
				behavior  monitoring  and  feedback  and  service  failure  recovery
				without  delay, etc.
			\end{itemize} 
			
			\item A integração de software e sistema é visível nos novos padrões. Por
			exemplo, colocar ambos na ISO 12207 é um recado direto de que o ciclo de
			vida do desenvolvimento de software deve considerar ambos.
		\end{itemize}
	}
\end{frame}


