\begin{frame}[parent={ie:agenda}, hasnext=true, hasprev=false]
	\frametitle{Qualidade de processo}
	
	\begin{block:concept}{Melhoria de Processo}
		Definição, implantação, medição, gerência, mudança e melhoria do
		processo de engenharia de software.
	\end{block:concept}
	
	\begin{block:fact}{Objetivos}
		\begin{itemize}
			\item Melhorar produtividade
			\begin{itemize}
				\item Diminuir retrabalho/Aumentar reúso
			\end{itemize}
			
			\item Previsibilidade
			\begin{itemize}
				\item Melhorar estimativas
				\item Padronizar qualidade do produto obtido
			\end{itemize}
			
			\item Controlar riscos
			
			\item Aumentar a qualidade do produto
		\end{itemize}
	\end{block:fact}
\end{frame}


\begin{frame}[hasnext=true, hasprev=true]
	\frametitle{Qualidade de processo}
	
	\begin{block:procedure}{Passos}
		\begin{itemize}
			\item Estabelecer infraestrutura para o processo.
			\item Planejar a implantação e alteração do processo.
			\item Implantar e alterar o processo.
			\item Avaliar o processo.
		\end{itemize}
	\end{block:procedure}

	\note{
		\begin{itemize}
			\item Infraestrutura:
			\begin{itemize}
				\item Recursos humanos, atribuição de responsabilidades
				\item Recursos financeiros
				\item Ferramentas
			\end{itemize}
			
			\item Geralmente é necessário ter um grupo de trabalho específico para
			tratar desta infraestrutura:
			\begin{itemize}
				\item Fábrica de experiências
			\end{itemize}
			
			\item Ao invés de começar com a definição de um novo processo, analise
			o processo atual!
			\begin{itemize}
				\item Mapeie as boas práticas
				\item Acrescente novas quando necessário
			\end{itemize}
		\end{itemize}
	}
\end{frame}